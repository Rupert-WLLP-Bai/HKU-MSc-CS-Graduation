\chapter{Conclusion}

In this study, we address the critical issue of entangled semantic embeddings, which has been a persistent challenge in click-through rate prediction tasks, and propose \textbf{Multi-faceted Semantic Disentanglement for CTR Prediction (MSD-CTR)} as a comprehensive solution. The main motivation behind our approach stems from the observation that traditional methods often fail to effectively separate and utilize different semantic aspects present in textual data, leading to suboptimal performance in CTR prediction scenarios. Our proposed MSD-CTR framework is designed to tackle this problem through a two-stage approach that first extracts disentangled representations and then effectively incorporates them into the prediction process.

Specifically, MSD-CTR utilizes \textit{DSTopic}, our novel disentangled semantic topic model, to extract and disentangle multi-faceted knowledge from textual information. The \textit{DSTopic} component operates through a carefully designed disentangled topic modeling framework that partitions the vocabulary into distinct subsets, ensuring that each semantic aspect is captured independently without interference from other aspects. This approach allows us to obtain clean, separated representations of different semantic dimensions present in the textual content. Building upon the disentangled representations learned by \textit{DSTopic}, MSD-CTR then employs \textit{TopicDRL} (Topic Guided Disentangled Representation Learning) to effectively incorporate the learned disentangled multi-faceted knowledge into the CTR prediction process. Rather than directly using the topic embeddings, \textit{TopicDRL} introduces carefully designed alignment losses that guide the learning process and ensure that the disentangled semantic information is properly integrated with the prediction objective.

To validate the effectiveness and generalizability of our proposed approach, we implement our method based on two well-established foundational CTR models, allowing us to demonstrate that MSD-CTR can enhance different baseline architectures. We conduct extensive experiments on four diverse CTR datasets that represent different domains and characteristics, ensuring comprehensive evaluation across various scenarios. The experimental results consistently demonstrate the effectiveness of the proposed method, showing significant improvements over baseline models and competing approaches across all tested datasets and base models. Furthermore, to provide deeper insights into the working mechanisms of our approach, we perform comprehensive qualitative and ablation studies. The qualitative studies validate the disentanglement capability of our method by examining the learned semantic aspects and demonstrating their interpretability and distinctiveness. The ablation studies systematically assess the impact of different components of MSD-CTR, including the contribution of \textit{DSTopic}, \textit{TopicDRL}, and various design choices within these modules, providing clear evidence of each component's importance to the overall performance improvement.